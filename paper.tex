\documentclass[aps,prd,twocolumn,superscriptaddress,groupedaddress]{revtex4}
% Intended to be built with pdflatex, using the REVTEX 4 package.
% Template taken from FermiLab APS Jouranl template.

% packages
\usepackage{amssymb}              % for \gtrsim ...
\usepackage{amsmath}              % for \text, align, split ...
\usepackage{color}                % for colorful margin notes
\usepackage[mathscr]{euscript}    % for \mathscr
\usepackage{graphicx}             % for \includegraphics
\usepackage{hyperref}             % for \url and \href

% commands/initialization
\hyphenation{ALPGEN} % avoid incorrect hyphenation
\hyphenation{EVTGEN} % ''
\hyphenation{PYTHIA} % ''

\newcommand{\todo}[1]{\marginpar{\tiny{\textcolor{red}{#1}}}}
%\renewcommand\todo[1]{} % if you wish to hide all todo notes

% *******************************************************************************
\begin{document}

\widetext
\leftline{Compiled \today}
\leftline{To be submitted to PRD}

\title{General Relativistic Ray Tracing for Neutrinos}

\author{M.\ Brett Deaton}
\affiliation{Joint Institute for Nuclear Astrophysics,
  Michigan State University, East Lansing, MI 48824, USA}
\affiliation{Department of Physics,
  North Carolina State University, Raleigh, NC 27695, USA}
\email{mbdeaton@ncsu.edu}

\author{Jerred Jesse}
\affiliation{Department of Physics \& Astronomy,
  Washington State University, Pullman, WA 99164, USA}

\author{Evan O'Connor}
\affiliation{Department of Physics,
  North Carolina State University, Raleigh, NC 27695, USA}

\author{Francois Foucart}
\affiliation{Lawrence Berkeley National Laboratory,
  1 Cyclotron Rd., Berkeley, CA 94720, USA}

\author{Andy Bohn}
\affiliation{Center for Radiophysics and Space Research,
  Cornell University, Ithaca, NY 14853, USA}

\author{Matthew D.\ Duez}
\affiliation{Department of Physics \& Astronomy,
  Washington State University, Pullman, WA 99164, USA}

\author{G.\ C.\ McLaughlin}
\affiliation{Department of Physics,
  North Carolina State University, Raleigh, NC 27695, USA}

% *******************************************************************************

\begin{abstract}
  We present a ray tracing algorithm for computing neutrino distributions in
  general numerical spacetimes with hydrodynamical sources.
  The formulation is fully coordinate invariant and accurately describes
  radiation in regimes spanning optically thick to optically thin.
  We test the new code, highlight its strengths and weaknesses, and
  apply it to estimate power
  liberated by neutrino-antineutrino annihilation in the funnel of a
  post-merger accretion disk.
\end{abstract}

\maketitle

Greek tensor indices ($\alpha, \beta, ...$) range over all four coordinates,
whereas Latin indices ($i, j, ...$) range over the spatial coordinates 1--3.
We use naturalized units in which $\hbar=c=1$.

\section{Ray tracing formalism}

The neutrino distribution function, $f(p_\beta; x^\alpha)$ is an invariant
quantity counting the number of neutrinos in a given momentum volume centered on
$p_\beta$ and spatial volume centered on $x^\alpha$.
The neutrino momentum is written
\begin{equation}
  \label{eqn:def_momentum}
  p_\beta = \varepsilon (u_\beta + \ell_\beta),
\end{equation}
with $u^\beta$ a general observer's velocity,
$\varepsilon=-p_\mu u^\mu$ the energy measured by that observer,
and $\ell_\beta$ the direction vector subject to the constraints
$u^\alpha \ell_\alpha = 0$ and
$\ell^\alpha \ell_\alpha=1$.
Because $\ell$ is normalized and perpendicular to the observer's velocity,
it has two degrees of freedom, which we can make explicit by defining its
spatial components with respect to spherical polar angles
\begin{equation}
  \label{eq:def_direction}
  \ell_i :=
  (\sin\upsilon_1\cos\upsilon_2,\sin\upsilon_1\sin\upsilon_2,\cos\upsilon_1).
\end{equation}

Neutrino radiation obeys the relativistic Boltzmann equation
\todo{integral isn't explicitly covariant}
\todo{correct this to differentiate in phase space}
\begin{equation}
  \label{eqn:boltzmann}
  p^\beta \nabla_\beta f = \mathscr{E} - \mathscr{K} f
  + \mathscr{K}_s \oint d\ell' \phi(\ell_\alpha,\ell_\beta') f,
\end{equation}
where $\nabla_\beta$ denotes a covariant derivative. We have introduced
$\mathscr{E}$, the invariant emissivity,
$\mathscr{K}\equiv\mathscr{K}_a+\mathscr{K}_s$, the invariant total opacity
comprised of absorption and scattering opacities, and
$\phi(\ell_\alpha,\ell_\beta')$, the elastic scattering kernel, representing the
likelihood of scattering from the direction $\ell_\beta'$ to the direction
$\ell_\alpha$, subject to the normalization
\todo{confirm this}
\begin{equation}
  \oint d\ell' \phi(\ell_\alpha,\ell_\beta') = 1.
  \nonumber
\end{equation}

\subsection{Trajectories}
\label{ssc:trajectories}
Each trajectory is labeled by a pair of vectors
giving some position on the trajectory, $x^\alpha$,
and the normalized momentum direction at that position,
$\ell_\beta$.
A family of trajectories shares the same $x^\alpha$; for example,
the family of emission trajectories is ($x^\alpha_e,\ell_\beta$),
and the family of observation trajectories is ($x^\alpha_o,\ell_\beta$).

A neutrino trajectory obeys the geodesic equations
\begin{equation}
\label{eqn:geodesic_x}
  \frac{d x^\alpha}{d\lambda} = p^\alpha,
\end{equation}
and
\begin{equation}
\label{eqn:geodesic_p}
  \frac{d p_\beta}{d\lambda} = -\psi_{\alpha\beta}\Gamma^\alpha_{\mu\gamma} p^\mu p^\gamma,
\end{equation}
where $\psi_{\alpha\beta}$ is the metric, and
$\Gamma^\alpha_{\mu\gamma}$ are the standard connection coefficients,
\begin{equation}
  \label{eqn:christoffel}
  \Gamma^\beta_{\alpha\gamma} =
  \frac{1}{2} \psi^{\beta\mu}
  (\psi_{\mu\alpha,\gamma} + \psi_{\mu\gamma,\alpha} - \psi_{\alpha\gamma,\mu}).
\end{equation}

Each trajectory is parameterized by proper distance, $s$, increasing in
the direction of $\ell$. We choose $s=0$ at $x^\alpha_e$.
and $s=s_o$ at $x^\alpha_o$.
According to Eqn.~\ref{eqn:geodesic_x} proper distance relates to the affine
parameter by $ds=p^t d\lambda$.

\section{Fundamental Tests}

\section{Neutrino annihilation}
The high disk mass model is described in \cite{deat2013-leakage}.

\bibliography{references}

\end{document}
