\documentclass[aps,prd,twocolumn,superscriptaddress,groupedaddress]{revtex4}
% Intended to be built with pdflatex, using the REVTEX 4 package.
% Template taken from FermiLab APS Jouranl template.

% packages
\usepackage{amssymb}              % for \gtrsim ...
\usepackage{amsmath}              % for \text, align, split ...
\usepackage{color}                % for colorful margin notes
\usepackage[mathscr]{euscript}    % for \mathscr
\usepackage{graphicx}             % for \includegraphics
\usepackage{hyperref}             % for \url and \href

% commands/initialization
\hyphenation{ALPGEN} % avoid incorrect hyphenation
\hyphenation{EVTGEN} % ''
\hyphenation{PYTHIA} % ''

\newcommand{\todo}[1]{\marginpar{\tiny{\textcolor{red}{#1}}}}
%\renewcommand\todo[1]{} % if you wish to hide all todo notes

% *******************************************************************************
\begin{document}

\widetext
\leftline{Compiled \today}
\leftline{To be submitted to PRD}

\title{General Relativistic Ray Tracing for Neutrinos}

\author{M.\ Brett Deaton}
\affiliation{Joint Institute for Nuclear Astrophysics,
  Michigan State University, East Lansing, MI 48824, USA}
\affiliation{Department of Physics,
  North Carolina State University, Raleigh, NC 27695, USA}
\email{mbdeaton@ncsu.edu}

\author{Jerred Jesse}
\affiliation{Department of Physics \& Astronomy,
  Washington State University, Pullman, WA 99164, USA}

\author{Evan O'Connor}
\affiliation{Department of Physics,
  North Carolina State University, Raleigh, NC 27695, USA}

\author{Francois Foucart}
\affiliation{Lawrence Berkeley National Laboratory,
  1 Cyclotron Rd., Berkeley, CA 94720, USA}

\author{Andy Bohn}
\affiliation{Center for Radiophysics and Space Research,
  Cornell University, Ithaca, NY 14853, USA}

\author{Matthew D.\ Duez}
\affiliation{Department of Physics \& Astronomy,
  Washington State University, Pullman, WA 99164, USA}

\author{G.\ C.\ McLaughlin}
\affiliation{Department of Physics,
  North Carolina State University, Raleigh, NC 27695, USA}

% *******************************************************************************

\begin{abstract}
  We present a ray tracing algorithm for computing high-resolution neutrino
  distributions in general numerical spacetimes with hydrodynamical sources.
  Our formulation is covariant and treats elastic scattering by incorporating
  estimates of the neutrino fields evolved in a low-order moment simulation.
  It handles radiation in regimes spanning optically thick to optically thin.
  We test the new code, highlight its strengths and weaknesses, and
  apply it to simulations of neutron star mergers to
  1) compute neutrino fluxes as a function of energy and emission angle, and
  2) compute the radiation pressure tensor in the funnel of the disk,
  demonstrating a way to improve the closure relations used in the truncated
  moment formalism.
\end{abstract}

\maketitle

\section{Introduction}
Neutrinos are one of the dominant energy transport phenomena at play in
neutron star mergers, heating, cooling, and pushing the disrupted nuclear
matter.
In addition, they change the composition of the matter via charged current
interactions.
Because neutrinos scatter over length scales both large and small with
respect to fluid scales, accurate models require a neutrino treatment that
respects the freedom of the neutrino distribution functions to vary
drastically from equilibrium.

This is a challenging task in the environment of a merger,
which generally lacks any spatial symmetries.
The state of the art today employs a truncated moment formalism
\citep{fouc2015-m1_bhns, fouc2016-m1_nsns},
evolving the zeroth- and first-angular moments of the distribution function
(the energy density and flux respectively) for the zeroth-energy moment
(the total energy).
Recently, \cite{fouc2016-m1_evolve_n} have expanded their code to also evolve
the number densities of neutrinos, giving a more accurate estimate of the
average neutrino energies.
But with only two angular moments and one energy moment, the distribution
function has very limited angular and spectral information.

\subsection{What physics problems do we want to address?}
But many interesting unsolved problems require an accurate model of the neutrino
spectra and angular distribution.
With a model of the neutrino emission from a merger we can
1) examine neutrino effects on the nucleosynthesis of the ejected material
\citep{robe2016-nsbh_sph_nucleo},
2) explore the rich flavor oscillation physics likely to occur here
\citep{malk2012-mnr_1, malk2014-mnr_2, malk2015-mnr_3, zhu2016-mnr_disk,
  vaan2016-uncovering_mnr},
3) improve closure relations used in truncated moment evolution schemes, and
4) look at jet formation due to neutrino annihilation
\citep{birk2007-nunubar}.

\subsection{Why ray tracing?}
In this work we present a ray tracing method to compute neutrino distribution
functions from the state of the art general relativistic radiation
hydrodynamics simulations.
We choose ray tracing because it is conceptually simple, numerically cheap,
and easily extends to high resolution in energy and angle.

\subsection{What is new/better about this ray tracing formulation?}
We build upon ray tracing methods already in the literature
\citep{hari2010-nunubar_gr, ?}
by including an elastic scattering approximation
\todo{and perhaps inelastic}
and removing any assumptions about the spacetime geometry.
To incorporate elastic scattering in our method, we employ the lower-order
estimates of the distribution function evolved in a moment scheme.

\subsection{Why does scattering matter?}
It is essential that we include scattering because although elastic scattering
doesn't change the emitted spectrum, it can significantly modify neutrino
distributions in angle, and dilute the emitted spectrum over a larger emitting
surface \citep{yonglin_collab}, especially for heavy-lepton neutrinos.
Any phenomena that involve neutrino-neutrino interactions (neutrino oscillations,
and neutrino-antineutrino annihilation) can depend sensitively on the angular
distribution.
And a diluted spectrum can have a different effect on the nucleosynthesis
occuring in the ejected and irradiated material.

\subsection{In what ways is this method limited?}
Ray tracing, however, has the fundamental limitation that it treats the
Boltzmann equation as completely decoupled between different neutrino momenta
and species. In this paper we seek to push this limitation by incorporating
coupling terms that depend on previously-evolved estimates of the neutrino
fields. This method is not a standalone radiation transport scheme, but
is presented as one component of a  hybrid scheme, piggybacking on a
lower-order radiation transport method as a post-processing step.

\subsection{Outline}
In Sec.~\ref{sec:formulation} we derive the ray tracing equations from the
Boltzmann equation and describe our numerical scheme.
In Sec.~\ref{sec:tests} we present some tests of the code.
In Sec.~\ref{sec:applications} we present two applications:
neutrino spectra as a function of observer's angle, and
a variable closure relation for a truncated moment formalism.
We present both applications in the dynamical environments following the merger
of a neutron star--neutron star and a black hole--neutron star binary.

Greek tensor indices ($\alpha, \beta, ...$) range over all four coordinates,
whereas Latin indices ($i, j, ...$) range over the spatial coordinates 1--3.
We use naturalized units in which $\hbar=c=1$.

\section{Ray tracing formulation}
\label{sec:formulation}
The neutrino distribution function, $f(x^\alpha; p_\beta)$ is an invariant
quantity counting the number of neutrinos in a given momentum volume centered on
$p_\beta$ and spatial volume centered on $x^\alpha$.
We may decompose the neutrino momentum
\begin{equation}
  \label{eqn:def_momentum}
  p_\beta = \varepsilon (u_\beta + \ell_\beta),
\end{equation}
with $u^\beta$ some fiducial observer's velocity,
$\varepsilon=-p_\mu u^\mu$ the energy measured by that observer,
and $\ell_\beta$ the direction subject to the constraints
$u^\alpha \ell_\alpha = 0$ and
$\ell^\alpha \ell_\alpha=1$.
With this decompositon we can write the arguments to the distribution function
$f(x^\alpha;\varepsilon,\ell_\beta)$.
Because $\ell_\beta$ is normalized and perpendicular to the observer's velocity,
it has two degrees of freedom, which we can make explicit by defining its
spatial components with respect to spherical polar angles
\begin{equation}
  \label{eq:def_direction}
  \ell_\alpha :=
  q (s,\sin\upsilon_1\cos\upsilon_2,\sin\upsilon_1\sin\upsilon_2,\cos\upsilon_1),
\end{equation}
with $q$ and $s$ functions of $\upsilon_1$ and $\upsilon_2$.

Neutrino radiation obeys the relativistic Boltzmann equation
\begin{equation}
  \label{eqn:boltzmann}
  \frac{d}{d\lambda}f(x^\alpha;\varepsilon,\ell_\beta) =
  C[f](x^\alpha;\varepsilon,\ell_\beta),
\end{equation}
where $\frac{d}{d\lambda}$ denotes a derivative with respect to the affine
parameter defining the neutrino momentum,
and $C[f](x^\alpha;\varepsilon,\ell_\beta)$
is the source term arising from interactions with the medium,
including absorption/emission, elastic scattering, inellastic scattering,
and pair annihilation/production processes.
We symbolize $C$ this way to remind us that it depends on the positon and
momentum of the neutrinos in question, and nonlocally on $f$ at other positions
and momenta.
We compute $C$ in App.~\ref{sec:source_terms}.

We make the right hand side of Eqn.~\ref{eqn:boltzmann} explicit by writing
the source term linear in $f$:
\begin{equation}
  \label{eqn:boltzmann_linear}
  \frac{d}{d\lambda}f =
  \mathscr{E}^* - \mathscr{K}^* f,
\end{equation}
where we have introduced
$\mathscr{E}^*$, the invariant emissivity, and
$\mathscr{K}^*$, the invariant opacity, and for simplicity we have suppressed
the arguments of $f$.
These coefficients account for Fermi-blocking and involve integrals over the
neutrino distribution function at other energies and angles,
$f(\varepsilon', \ell'_\beta)$.
We derive them for the relevant microphysical processes in
App.~\ref{sec:source_terms}

\subsection{Trajectories}
\label{ssc:trajectories}
Each trajectory is labeled by a pair of vectors
giving some position on the trajectory, $x^\alpha$,
and the normalized momentum direction at that position,
$\ell_\beta$.
A family of trajectories shares the same $x^\alpha$; for example,
the family of emission trajectories is ($x^\alpha_e,\ell_\beta$),
and the family of observation trajectories is ($x^\alpha_o,\ell_\beta$).

A neutrino trajectory obeys the geodesic equation, which may be decomposed into
the coupled first-order equations
\begin{equation}
\label{eqn:geodesic_x}
  \frac{d x^\alpha}{d\lambda} = p^\alpha,
\end{equation}
and
\begin{equation}
\label{eqn:geodesic_p}
  \frac{d p_\beta}{d\lambda} = -\Gamma^\alpha_{\mu\beta} p^\mu p_\alpha,
\end{equation}
where $\Gamma^\alpha_{\mu\gamma}$ are the standard connection coefficients,
\begin{equation}
  \label{eqn:christoffel}
  \Gamma^\beta_{\alpha\gamma} =
  \frac{1}{2} \psi^{\beta\mu}
  (\psi_{\mu\alpha,\gamma} + \psi_{\mu\gamma,\alpha} - \psi_{\alpha\gamma,\mu}),
\end{equation}
and $\psi_{\alpha\beta}$ is the spacetime metric.

Each trajectory is parameterized by affine parameter, $\lambda$, increasing in
the direction of $\ell$. We choose $\lambda=0$ at $x^\alpha_e$.
and $\lambda=\lambda_o$ at $x^\alpha_o$.
According to Eqn.~\ref{eqn:geodesic_x} proper distance relates to the affine
parameter by $ds=p^t d\lambda$.

\subsection{Numerical Implementation}
\label{sec:numerical}

\section{Fundamental Tests}
\label{sec:tests}

\section{Applications}
\label{sec:applications}

\subsection{Emission Spectra}
\label{ssec:spectra}
The merger models are described in \cite{fouc2015-m1_nsbh,fouc2016-m1_nsns}.

\subsection{Closure Relation}
\label{ssec:closure}

% ******************************************************************************
\appendix

\section{Source Terms}
\label{sec:source_terms}
Here we present the source terms comprising the right hand side of the
Boltzmann Equation (Eqn.~\ref{eqn:boltzmann}) in terms of the microphysical
processes of absorption/emission, elastic scattering, inellastic scattering,
and pair annihilation/production.

% ******************************************************************************
\bibliography{references}

\end{document}
