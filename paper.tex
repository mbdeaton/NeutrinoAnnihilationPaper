\documentclass[aps,prd,twocolumn,superscriptaddress,groupedaddress]{revtex4-1}
% Intended to be built with pdflatex, using the REVTEX 4 package.
% Template taken from FermiLab APS Jouranl template.

% packages
\usepackage{amssymb}              % for \gtrsim ...
\usepackage{amsmath}              % for \text, align, split ...
\usepackage{color}                % for colorful margin notes
\usepackage[mathscr]{euscript}    % for \mathscr
\usepackage{graphicx}             % for \includegraphics
\usepackage{hyperref}             % for \url and \href
\usepackage{tabularx}             % for tabularx

% commands/initialization
\hyphenation{ALPGEN} % avoid incorrect hyphenation
\hyphenation{EVTGEN} % ''
\hyphenation{PYTHIA} % ''

\newcommand{\todo}[1]{\marginpar{\tiny{\textcolor{red}{#1}}}}
%\renewcommand\todo[1]{} % if you wish to hide all todo notes

% *******************************************************************************
\begin{document}

\widetext
\leftline{Compiled \today}
\leftline{To be submitted to PRD}

\title{Scattering and Pair Processes in General Relativistic Ray Tracing for Neutrinos}
\todo{you sure about including all processes?}

\author{M.\ Brett Deaton}
\affiliation{Joint Institute for Nuclear Astrophysics,
  Michigan State University, East Lansing, MI 48824, USA}
\affiliation{Department of Physics,
  North Carolina State University, Raleigh, NC 27695, USA}
\email{mbdeaton@ncsu.edu}

\author{Jerred Jesse}
\affiliation{Department of Physics \& Astronomy,
  Washington State University, Pullman, WA 99164, USA}

\author{Evan O'Connor}
\affiliation{Department of Physics,
  North Carolina State University, Raleigh, NC 27695, USA}

\author{Francois Foucart}
\affiliation{Lawrence Berkeley National Laboratory,
  1 Cyclotron Rd., Berkeley, CA 94720, USA}

\author{Andy Bohn}
\affiliation{Center for Radiophysics and Space Research,
  Cornell University, Ithaca, NY 14853, USA}

\author{Matthew D.\ Duez}
\affiliation{Department of Physics \& Astronomy,
  Washington State University, Pullman, WA 99164, USA}

\author{G.\ C.\ McLaughlin}
\affiliation{Department of Physics,
  North Carolina State University, Raleigh, NC 27695, USA}

% *******************************************************************************

\begin{abstract}
  We present a covariant ray tracing algorithm for computing high-resolution
  neutrino distributions in general numerical spacetimes with hydrodynamical
  sources.
  Our formulation treats scattering and pair processes
  by incorporating estimates of the background neutrino fields.
  These background fields may be taken from a low-order moment simulation,
  be supplied analytically in simple cases,
  or be ignored entirely, in which case the method
  reduces to a standard state-of-the-art ray tracing formulation.
  It handles radiation in regimes spanning optically thick to optically thin.
  We test the new code, highlight its strengths and weaknesses, and
  apply it to simulations of neutron star mergers to
  1) compute neutrino fluxes as a function of energy and emission angle, and
  2) compute the radiation pressure tensor in the funnel of the disk,
  demonstrating a way to improve the closure relations used in the truncated
  moment formalism.
\end{abstract}

\maketitle

\section{Introduction}
Neutrinos are one of the dominant energy transport phenomena at play in
neutron star mergers, heating, cooling, and pushing the disrupted nuclear
matter.
In addition, they change the composition of the matter via charged current
interactions.
Because neutrinos scatter over length scales both large and small with
respect to fluid scales, accurate models require a neutrino treatment that
respects the freedom of neutrino distribution functions to vary
drastically from thermodynamic equilibrium.

This is a challenging task in the environment of a merger,
which generally lacks any spatial symmetries so that fully general
seven-dimensional solutions to the Boltzmann Equation are not feasible.
Leakage approximations
\citep{deat2013-leakage, pere2016-asl, radi2016-dynamical, ?}
capture the qualitative effects of neutrinos on the matter, but
produce limited information about the neutrino field itself.
Monte Carlo methods \citep{abdi2012-monte_carlo, rich2015-monte_carlo}
suffer from noise.
\todo{discuss mc more clearly}
The state of the art today employs a truncated moment formalism
\citep{shib2011-truncated_moment, fouc2015-m1_nsbh, fouc2016-m1_nsns,
  just2015-m1_code, ?},
evolving the zeroth- and first-angular moments of the distribution function
(the energy density and flux respectively) representing the zeroth-energy moment
of the radiation field (the total energy).
Recently, \cite{fouc2016-m1_evolve_n} have expanded their code to also evolve
the number densities of neutrinos, providing an estimate of the average neutrino
energies in addition to their totals.
But with only two angular moments and one energy moment, the distribution
function has very limited angular and spectral information.

\subsubsection*{What physics problems do we want to address?}
But many interesting unsolved problems require an accurate model of the neutrino
spectra and angular distributions.
With a model of the neutrino emission from a merger we can
1) examine neutrino effects on the nucleosynthesis of the ejected material
\citep{surm2011-nickel_56, robe2016-sph_nu_nucleo},
2) explore the rich flavor oscillation physics likely to occur
\citep{malk2012-mnr_1, malk2015-mnr_2, malk2016-mnr_3, zhu2016-mnr_nsns_remnant,
  vaan2016-uncovering_mnr},
3) improve closure relations used in truncated moment schemes
\citep{ramp2002-truncated_moment, shib2011-truncated_moment,
  card2013-truncated_moment, fouc2015-m1_nsbh, ocon2015-gr1d_with_nu}, and
4) study possible jet formation due to neutrino annihilation
\citep{ruff1999-nunubar_nsns, asan2000-nunubar, birk2007-nunubar,
  hari2010-gr_nunubar_collapsar, zala2011-nunubar, leng2014-nunubar}.

\subsubsection*{Why ray tracing?}
In this work we present a ray tracing method to compute neutrino distribution
functions from state-of-the-art general relativistic radiation
hydrodynamics simulations.
We choose ray tracing because it is conceptually simple, numerically cheap,
and easily extends to high resolution in energy and angle.

With a ray tracing method we approach radiation transport from the perspective
of a single observer at a spacetime event $x_o^\alpha$.
Our goal is to compute the distribution function $f(x_o^\alpha;p_\beta)$,
or the amount of radiation with momentum $p_\beta$ impinging on $x_o^\alpha$.
To do so we trace a geodesic trajectory in the backwards direction
$-p_\beta$ to sample the incoming radiation along that line of sight.
By tracing a family of rays intersecting $x_o^\alpha$ we build up a
picture of the distribution function there.
And by sampling many observation points we construct a global picture
of $f(x^\alpha;p_\beta)$.

The ray tracing framework
is conceptually simple because it interprets the equation of radiation
transport (Eqn.~\ref{eqn:boltzmann})
as a one-dimensional ordinary differential equation,
it is numerically cheap because it confines computations of
$f(x_o^\alpha;p_\beta)$ to the past light-cone of $x_o^\alpha$,
and it easily extends to high resolution in energy and angle by simply
increasing the number of rays sampled.

\subsubsection*{What is new and better about this ray tracing formulation?}
Several ray tracing formulations for radiation transport already exist.
Most formulations assume an analytic spacetime metric
\citep{birk2007-nunubar, hari2010-gr_nunubar_collapsar, kova2011-gr_ray_tracing}.
And many make the simplifying assumption of blackbody emission from a
neutrinosurface \citep{birk2007-nunubar, kova2011-gr_ray_tracing}.
Current state-of-the-art ray tracing formulations avoid the assumption of
blackbody spectra by integrating a local emissivity along each geodesic
(e.g. \cite{hari2010-gr_nunubar_collapsar} for neutrinos and
\cite{youn2012-gr_radiative_transfer} for photons).
But no formulations to date account for the important scattering and pair
processes outlined in Tab.~\ref{tab:neutrino_processes}.

We build upon existing ray tracing formulations
by eschewing any assumptions about the spacetime geometry
and by including scattering and pair processes in the integration along each
geodesic.
To incorporate these neutrino processes in our method, we employ
estimates of the first two angular moments of the distribution function
computed in a truncated moment scheme (formulated by
\cite{thor1981-truncated_moment, shib2011-truncated_moment}
and applied e.g. in \cite{ramp2002-truncated_moment,
  card2013-truncated_moment, fouc2015-m1_nsbh, ocon2015-gr1d_with_nu}).
If the first two angular moments are not available either from
a truncated moment evolution or a trustworthy analytical estimate,
our method reduces to the current state-of-the-art ray tracing method,
neglecting scattering and pair processes.

\subsubsection*{Why does a covariant formulation matter?}
We formulate our ray tracing equations covariantly---free from
assumptions about the spacetime geometry or coordinates. This is essential
because we want to apply the method as a postprocessing step using
time snapshots of data computed from general relativistic evolutions.
The spacetime represented in these snapshots is not analytic (i.e. Kerr).
And even in configurations that are described well by the Kerr metric
(e.g. a low-mass disk around a massive black hole),
the evolution coordinates are unlikely to present the metric in
a familiar analytic form.
This is because integrating the Einstein Equations often requires complicated,
time-dependent guage conditions
\citep{lind2007-gen_harmonic, fouc2013-compactness_and_spin}.

\subsubsection*{Why do scattering and pair processes matter?}
Inelastic scattering and pair processes introduce significant changes to
a spectrum,
\todo{clarify these terms}
\todo{justify these statements}
and elastic scattering can signicantly modify neutrino
distributions in angle, and dilute the emitted spectrum over a larger emitting
surface \citep{pere2016-asl}, especially for heavy-lepton neutrinos.
Any phenomena that involve neutrino-neutrino interactions (neutrino oscillations,
and neutrino-antineutrino annihilation) can depend sensitively on the angular
distribution.
And spectral changes can strongly affect the nucleosynthesis
occuring in the ejected and irradiated material.
\todo{justify these statements}

\subsubsection*{In what ways is this method limited?}
Ray tracing is ideally suited to problems requiring detailed knowledge of
radiation distribution functions over small regions of spacetime:
for example along a matter or radiation trajectory,
or over a small volume outside a source.
But the method becomes computationally burdensome when the inquiry extends
to large volumes of spacetime.

More fundamentally, though, ray tracing is limited by its naive
treatment of the Boltzmann Equation (Eqn.~\ref{eqn:boltzmann}),
a treatment which essentially decouples different neutrino momenta and species.
\todo{clarify how momenta are decoupled}
In this paper we outwit this limitation by incorporating coupling
terms that depend on either previously-evolved or analytical
estimates of the neutrino fields.
We also retain the freedom to drop the coupling terms, in which case our method
becomes similar to existing state-of-the-art ray tracing methods.
Our method is not a standalone radiation transport scheme,
but serves as the final component of a hybrid scheme,
piggy-backing on a lower-order radiation transport method as a
post-processing step.

\subsection{Outline}
In Sec.~\ref{sec:formulation} we derive the ray tracing equations from the
Boltzmann Equation and describe our numerical scheme.
In Sec.~\ref{sec:tests} we present some tests of the code.
In Sec.~\ref{sec:applications} we present two applications:
neutrino spectra as a function of observer's angle, and
a variable closure relation for a truncated moment formalism.
We present both applications in the dynamical environments following the merger
of a neutron star--neutron star and a black hole--neutron star binary.

Greek tensor indices ($\alpha, \beta, ...$) range over all four coordinates,
whereas Latin indices ($i, j, ...$) range over the spatial coordinates 1--3.
We use naturalized units in which $\hbar=c=1$.

\section{Ray tracing formulation}
\label{sec:formulation}
The neutrino distribution function, $f(x^\alpha; p_\beta)$ is an invariant
quantity counting the number of neutrinos in a given momentum volume centered on
$p_\beta$ and spatial volume centered on $x^\alpha$.
We may decompose the neutrino momentum
\begin{equation}
  \label{eqn:def_momentum}
  p_\beta = \varepsilon (u_\beta + \ell_\beta),
\end{equation}
with $u^\beta$ some fiducial observer's velocity,
$\varepsilon=-p_\mu u^\mu$ the energy measured by that observer,
and $\ell_\beta$ the direction normal subject to the constraints
$u^\alpha \ell_\alpha = 0$ and $\ell^\alpha \ell_\alpha=1$.
With this decompositon we can write the arguments to the distribution function
$f(x^\alpha;\varepsilon,\ell_\beta)$.
Because $\ell_\beta$ is subject to two constraints
(normalization and orthogonality to the observer's velocity)
it has two degrees of freedom; we this make explicit by defining its
spatial components with respect to spherical polar angles
\begin{equation}
  \label{eq:def_direction}
  \ell_\alpha :=
  q (s,\sin\upsilon_1\cos\upsilon_2,\sin\upsilon_1\sin\upsilon_2,\cos\upsilon_1),
\end{equation}
with $q$ and $s$ functions of $\upsilon_1$ and $\upsilon_2$.
Now our symbol for the distribution function,
$f(x^\alpha;\varepsilon,\upsilon_1,\upsilon_2)$,
makes manifest its seven independent arguments.

Neutrino radiation obeys the relativistic Boltzmann Equation
\todo{explain limit of QKEs}
\begin{equation}
  \label{eqn:boltzmann}
  \frac{d}{d\lambda}f(x^\alpha;\varepsilon,\upsilon_1,\upsilon_2) = C[f],
\end{equation}
where $\frac{d}{d\lambda}$ denotes a derivative with respect to the affine
parameter defining the neutrino momentum (Eqn.~\ref{eqn:geodesic_x})
and $C[f]$ is the source term arising from interactions with the medium.
The source term varies over phase space $(x^\alpha,p_\beta)$,
and depends locally on the distribution function
$f(x^\alpha;\varepsilon,\upsilon_1,\upsilon_2)$,
positon $x^\alpha$, and momentum $p_\beta$ of the neutrinos in question,
and nonlocally on the distribution function of this neutrino and its
antiparticle at different momenta, $f'$ and $\bar{f}'$.
For clarity we symbolize all of these dependencies with the shorthand $C[f]$.
The various neutrino interactions contributing to $C[f]$ are detailed in
App.~\ref{sec:source_terms}.

We make the right hand side of Eqn.~\ref{eqn:boltzmann} explicit by writing
the source term linear in $f$:
\begin{equation}
  \label{eqn:boltzmann_linear}
  \frac{d}{d\lambda}f =
  \mathscr{E}^* - \mathscr{K}^* f,
\end{equation}
where we have introduced
$\mathscr{E}^*$, the invariant total emissivity, and
$\mathscr{K}^*$, the invariant total opacity,
and for simplicity we have suppressed the arguments of $f$.
These coefficients are derived by considering Fermi-blocking and
dependencies on neutrino and antineutrino distribution functions at other
momenta.
They are derived in App.~\ref{sec:source_terms}.

\subsection{Trajectories}
\label{ssec:trajectories}
Each trajectory is labeled by a pair of vectors
giving some event on the trajectory, $x^\alpha$,
and the momentum at that event, $p_\beta$.
A family of trajectories shares the same $x^\alpha$; for example,
the family of emission trajectories is ($x^\alpha_e,p_\beta$),
and the family of observation trajectories is ($x^\alpha_o,p_\beta$).

Neutrino trajectories obey the geodesic equation, which may be decomposed into
the coupled first-order equations
\begin{equation}
\label{eqn:geodesic_x}
  \frac{d x^\alpha}{d\lambda} = p^\alpha,
\end{equation}
and
\begin{equation}
\label{eqn:geodesic_p}
  \frac{d p_\beta}{d\lambda} = -\Gamma^\alpha_{\mu\beta} p^\mu p_\alpha,
\end{equation}
where $\Gamma^\alpha_{\mu\gamma}$ are the standard connection coefficients,
\begin{equation}
  \label{eqn:christoffel}
  \Gamma^\beta_{\alpha\gamma} =
  \frac{1}{2} \psi^{\beta\mu}
  (\psi_{\mu\alpha,\gamma} + \psi_{\mu\gamma,\alpha} - \psi_{\alpha\gamma,\mu}),
\end{equation}
and $\psi_{\alpha\beta}$ is the spacetime metric.

Each trajectory is parameterized by affine parameter, $\lambda$, increasing in
the direction of $\ell_\beta$. We choose $\lambda=0$ at $x^\alpha_e$
and $\lambda=\lambda_o$ at $x^\alpha_o$.
According to Eqn.~\ref{eqn:geodesic_x} proper distance relates to the affine
parameter by $ds=p^t d\lambda$.
\todo{show}

\subsection{The Rendering Equation}
\label{ssec:rendering_eqn}

\subsection{Neutrinosurface approximation}
\label{ssec:neutrinosurface}

\subsection{Moments of the Distribution Function}
\label{ssec:moments}

\subsection{Numerical Implementation}
\label{ssec:numerical}

\section{Fundamental Tests}
\label{sec:tests}

\subsection{Dummy Tests}
Maybe we don't need to present these tests, but make sure to establish
confidence in the basics:
gravitational redshift, Doppler shift, thermodynamic equilibrium.

\subsection{Homogenous Absorbing Star}
Test described in~\cite[Sec.~3.2]{smit1997-two_moment}

\subsection{Homogenous Scattering Star}
Test presented in~\cite{humm1971-grey_transfer}
and described in~\cite[Sec.~9.1.1]{abdi2012-monte_carlo}.

\subsection{Post-Bounce Neutron Core}
Standard radiation transport test described in~\cite{ocon2015-gr1d_with_nu},
and in \cite[App.~E.6]{fouc2015-m1_nsbh},
and in~\cite{abdi2012-monte_carlo}.

\subsection{Disk Comparison to M1}
Compute total and relative number and energy luminosities for a disk evolved
in one of~\cite{fouc2015-m1_nsbh, fouc2016-m1_nsns, fouc2016-m1_evolve_n}.
Expect agreement within factors of a few (?).

\section{Applications}
\label{sec:applications}

\subsection{Emission Spectra}
\label{ssec:spectra}
The merger models are described in \cite{fouc2015-m1_nsbh, fouc2016-m1_nsns}.
Something like this is presented in~\cite[Fig.~10-11]{pere2014-nu_wind};
compare.

\subsection{Closure Relation}
\label{ssec:closure}
This is a little tricky. Unless we can come up with a new improved analytic
closure relation and use ray tracing to evaluate its trustworthiness,
it's going to be hard to get anything useful from computing
$P^{\alpha\beta}$.

Maybe the variable Eddington closure is more suitable to this application.
If I understand it right, applying ray tracing to that method would require
computing $P^{\alpha\beta}$ on the fly throughout the spatial volume.
That would certainly be doable for a quasi-stationary disk configuration.

% ******************************************************************************
\appendix

\section{Source Terms}
\label{sec:source_terms}
Here we present the source terms comprising the right hand side of the
Boltzmann Equation (Eqn.~\ref{eqn:boltzmann}) in terms of the microphysical
processes of absorption/emission, elastic scattering, inellastic scattering,
and pair annihilation/production
(detailed in Tab.~\ref{tab:neutrino_processes}).

\begin{table}
  \centering
  \begin{tabularx}{\columnwidth}{X X}
    \hline \hline
    absorption/emission
    & $\nu_e + n \leftrightarrow e^- + p$                          \\
    & $\bar{\nu}_e + p \leftrightarrow e^+ + n$                    \\
    & $\nu_e + {}^AZ \leftrightarrow e^- + {}^A(Z+1)$              \\
    & $\bar{\nu}_e + {}^AZ \leftrightarrow e^+ + {}^A(Z-1)$        \\
    \hline
    elastic scattering
    & $\nu + N \leftrightarrow \nu + N$                            \\
    & $\nu + {}^AZ \leftrightarrow \nu + {}^AZ$                    \\
    \hline
    inelastic scattering
    & $\nu + e^- \leftrightarrow \nu' + e^{-'}$                    \\
    & $\nu + e^+ \leftrightarrow \nu' + e^{+'}$                    \\
    \hline
    pair processes
    & $\nu + \bar{\nu} \leftrightarrow e^{-} + e^{+}$              \\
    & $\nu + \bar{\nu} + N + N \leftrightarrow N' + N'$            \\
    & $\nu + \bar{\nu} \leftrightarrow \gamma$                     \\
    \hline \hline
  \end{tabularx}
  \caption{
    We analyze neutrino interaction processes in terms of these categories.
    $\nu$ without a label represents a neutrino or antineutrino of any flavor,
    $N$ represents a nucleon $n$ or $p$,
    ${}^ZA$ represents a nucleus with mass number $A$ and charge $Z$, and
    $\gamma$ represents a high-energy photon.
    A prime indicates a change in that particle's energy.
  }
  \label{tab:neutrino_processes}
\end{table}

% ******************************************************************************
\bibliography{references}

\end{document}
